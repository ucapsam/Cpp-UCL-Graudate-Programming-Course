\documentclass{beamer}
%\documentclass[handout]{beamer}

% Input all common stuff


% THEME SETTINGS %%%%%%%%%%%%
\usetheme[secheader]{Boadilla}
\usecolortheme{default}
\useinnertheme{circles}
\setbeamertemplate{blocks}[default]

% Get rid of bottom navigation bars
\setbeamertemplate{footline}[page number]{}

% Gets rid of navigation symbols
\setbeamertemplate{navigation symbols}{}

% PACKAGES %%%%%%%%%%%%%%%%%%%
\usepackage{listings}
\usepackage{xcolor}
\usepackage{multirow}
\usepackage{array}
\usepackage{verbatim}
\usepackage{pbox}
\usepackage{tabularx}

% LISTINGS STYLE %%%%%%%%%%%%%
\definecolor{bluekeywords}{rgb}{0.13,0.13,1}
\definecolor{greencomments}{rgb}{0,0.5,0}
\definecolor{redstrings}{rgb}{0.9,0,0}
\definecolor{listingsbg}{HTML}{EFEFFF}

\lstset{language=C++,
  showspaces=false,            % show spaces adding particular underscores
  showtabs=false,              % show tabs within strings adding particular underscores
  breaklines=true,             % sets automatic line breaking
  showstringspaces=false,      % underline spaces within strings
  breakatwhitespace=true,      % sets if automatic breaks should only happen at whitespace
  captionpos=b,                % sets the caption-position to bottom
  escapeinside={(*@}{@*)},
  commentstyle=\color{greencomments},
  keywordstyle=\color{bluekeywords}\bfseries,
  stringstyle=\color{redstrings},
  basicstyle=\ttfamily\fontsize{9}{10}\selectfont,
  backgroundcolor=\color{listingsbg}
}

% MACROS %%%%%%%%%%%%%%%%%%%%%
\newcommand{\lstsrctitle}[0]{%
 {\scriptsize\lstname}
}

\newcommand{\kw}[1]{\texttt{\textcolor{bluekeywords}{\ttfamily\fontsize{9}{10}\selectfont #1}}}

\newenvironment{varblock}[3]{%
  \setbeamercolor{block body}{#2}
  \setbeamercolor{block title}{#3}
  \begin{block}{#1}}{\end{block}}

% Colours taken from http://twitter.github.com/bootstrap/components.html#alerts
% Do block %%%%%%%%%%%%%%%%%%%%%%
\definecolor{doblocktext}{HTML}{468847}
\definecolor{doblockbg}{HTML}{DFF0D8}
\newenvironment{doblocke}[0]{% begin
  \begin{varblock}{\textbf{Do}}%
  {bg=doblockbg,fg=doblocktext}{bg=doblocktext,fg=white}%
  \setbeamercolor{itemize item}{fg=doblocktext}
  \lstset{backgroundcolor=}% The background should be the same as that of the block
  } % begin
  {\end{varblock}} %end
  
\newcommand{\doblock}[1]{%
 \begin{doblocke}#1\end{doblocke} 
}

% Don't block %%%%%%%%%%%%%%%%%%%
\definecolor{dontblocktext}{HTML}{B94A48}
\definecolor{dontblockbg}{HTML}{F2DEDE}  
\newenvironment{dontblocke}[0]{% begin
  \begin{varblock}{\textbf{Don't}}%
  {bg=dontblockbg,fg=dontblocktext}{bg=dontblocktext,fg=white}
  \setbeamercolor{itemize item}{fg=dontblocktext}
  \lstset{backgroundcolor=}% The background should be the same as that of the block
  }%
  {\end{varblock}} %end

\newcommand{\dontblock}[1]{%
 \begin{dontblocke}#1\end{dontblocke} 
}

% Defi block %%%%%%%%%%%%%%%%%%%%
\definecolor{defiblocktext}{HTML}{3A87AD}
\definecolor{defiblockbg}{HTML}{D9EDF7}
\newenvironment{defiblocke}[1]{% begin
  \begin{varblock}{\textbf{Definition}}%
  {bg=defiblockbg,fg=defiblocktext}{bg=defiblocktext,fg=white}
  \lstset{backgroundcolor=}% The background should be the same as that of the block
  \textit{#1} }
  {\end{varblock}} %end

\newcommand{\defiblock}[2]{%
 \begin{varblock}{\textbf{Definition}}%
 {bg=defiblockbg,fg=defiblocktext}{bg=defiblocktext,fg=white}%
  \begin{tabularx}{\linewidth}{lX}\textit{#1} & #2\end{tabularx}
 \end{varblock} 
}

\newenvironment{defiblockbaree}[1]{ %begin
  \begin{varblock}{\textbf{#1}}%
  {bg=defiblockbg,fg=defiblocktext}{bg=defiblocktext,fg=white}
  \lstset{backgroundcolor=}} %begin
  {\end{varblock}} %end

\definecolor{warnblocktext}{HTML}{C09853}
\definecolor{warnblockbg}{HTML}{FCF8E3}
\newenvironment{warnblocke}[0]{% begin
  \begin{varblock}{\textbf{Warning!}}%
  {bg=warnblockbg,fg=warnblocktext}{bg=warnblocktext,fg=white}
  \lstset{backgroundcolor=}% The background should be the same as that of the block
  \setbeamercolor{itemize item}{fg=warnblocktext}
  \setbeamercolor{item projected}{fg=white,bg=warnblocktext}
  } % begin
  {\end{varblock}} % end

\newcommand{\warnblock}[1]{%
  \begin{warnblocke}#1\end{warnblocke}
} % newcommand

\newcommand{\cout}[1]{%
 Output: \pbox[t]{\textwidth}{\ttfamily\fontsize{9}{10}\selectfont{}#1}
}

\newenvironment{doitemize}[0]{%
  \begin{itemize}}%
  {\end{itemize}} %end


% CONSTANTS %%%%%%%%%%%%%%%%%%%%%%%%%%%

%Symbol for backslash in texttt \char`\\

% Common title slide stuff
\title{Introduction to Scientific Programming with C++}
\author{Martin Uhrin}
\institute{UCL}
\date{February 11-13th 2013}

\subtitle{Session 4: Input, output and text manipulation}


\begin{document}

\frame{\titlepage}

\begin{frame}
\frametitle{Table of Contents}
\tableofcontents
\end{frame}

\section{Output}

\begin{frame}[fragile]
  \frametitle{Writing to a file}
  
  Let's say we wanted to gift someone with an ASCII caterpillar, here's how:
  \pause
  \lstinputlisting[language=C++,%
  	aboveskip=0pt,belowskip=0pt,
  	title=\lstsrctitle]%
  	{../code/4_io/lectures/simple_output.cpp}

  \pause Doesn't look like much does it?  That's because I've had to use escape characters to print some of the symbols:
  	\texttt{\textbackslash} is \texttt{\textbackslash\textbackslash} and
  	\texttt{"}  is \texttt{\textbackslash"} \footnote{For full list see \url{http://en.cppreference.com/w/cpp/language/escape}}

\end{frame}

\begin{frame}[fragile]
  \frametitle{Introduction to  \texttt{ofstream}}
  \framesubtitle{\textbf{O}utput \textbf{f}ile \textbf{stream}}
  All of the action happens with the \texttt{ofstream} object.  Let's have a closer look.
  \pause
  \begin{block}{Constructing \texttt{ofstream}}
    \begin{lstlisting}[aboveskip=0pt]
  ofstream();
  ofstream(const char * filename, ios_base::openmode mode = ios_base::out);
    \end{lstlisting}
  Two ways to construct an ofstream.  Second is equivalent to:
    \begin{lstlisting}
  std::ofstream myFile;
  myFile.open(filename, std::ios_base::out);
    \end{lstlisting}
    The \texttt{openmode} tells \texttt{ofstream} how you want to open the file\footnote{See \url{http://en.cppreference.com/w/cpp/io/ios_base/openmode} for a full list.}.
  \end{block}

\end{frame}

\begin{frame}[fragile]
  \frametitle{Introduction to \texttt{ofstream}}
  \begin{block}{Opening a file stream}
  	\begin{lstlisting}[aboveskip=0pt]
  void open(const char * filename, ios_base::openmode mode = ios_base::out);
  	\end{lstlisting}
		Does what it says on the tin: opens a file called \texttt{filename} with mode \texttt{mode}.
  
  \end{block}
  \pause
  \begin{block}{Writing to a file: the insertion operator}
  \begin{lstlisting}[aboveskip=0pt]
  ostream & operator <<(/*..stuff..*);
  \end{lstlisting}
  Use this to operator to insert text, numbers, strings and other things\footnote{See \url{http://en.cppreference.com/w/cpp/io/basic_ostream/operator_ltlt} for full list.} into a file.
  \end{block}
  
\end{frame}

\begin{frame}[fragile]
  \frametitle{Introduction to \texttt{ofstream}}
  
  \begin{block}{Closing a file}
  \begin{lstlisting}[aboveskip=0pt]
  void close();
  \end{lstlisting}
  Tells the operating system to close the file because we no longer need it.
  \end{block}
  \pause
  \begin{warnblocke}
    \begin{itemize}
      \item{You do not need to close files, the fstream destructor will do it for you when you leave the current scope.}
      \pause
      \item{If you want to close a file manually check to see if the file is open before trying to close it:
      \begin{lstlisting}
  if(myFile.is_open()){
    myFile.close();
  }
      \end{lstlisting}
      Calling close on a file that is not open is an error!}
    \end{itemize}
  \end{warnblocke}
  
\end{frame}

\section{Input}

\begin{frame}[fragile]
	\frametitle{Reading from a file}
	
	Reading from a text file is relatively easy.  Let's dive in:
	\begin{columns}[t]
	  \begin{column}[T]{0.5\textwidth}
  		\lstinputlisting[language=C++,%
    		basicstyle=\ttfamily\fontsize{7}{8}\selectfont{},%
    		title=\lstsrctitle,%
    		aboveskip=0pt,belowskip=0pt]%
    		{../code/4_io/lectures/simple_input.cpp}
		\end{column}
		\pause
		\begin{column}[T]{0.5\textwidth}
		  Output:
			\begin{verbatim}Here, have a caterpillar:
						
   \
    '-.__.-'
    /oo |--.--,--,--.
    \_.-'._i__i__i_.'
          """""""""
		\end{verbatim}
		\end{column}
	\end{columns}
\end{frame}

\begin{frame}[fragile]
  \frametitle{Introduction to \texttt{ifstream}}
  
  The \texttt{ifstream} (input file stream) class is where most of the magic happened.  Let's take a closer look:
  \pause
  \begin{block}{Constructing ifstream}
  	\begin{lstlisting}[aboveskip=0pt]
  ifstream();
  ifstream(const char * filename, ios_base::openmode mode = ios_base::in);
		\end{lstlisting}
		To create one we give it a file name.  Second argument tells \texttt{ifstream} how to open the file.
  		\lstinputlisting[language=C++,%
    		belowskip=0pt, linerange={8-8}]%
    		{../code/4_io/lectures/simple_input.cpp}
  \end{block}
  \pause
  \begin{block}{Checking if the file is ready to be read from}
  	\begin{lstlisting}[aboveskip=0pt]
  bool good() const;
  	\end{lstlisting}
  	If there has been an error or the end of file has been reached, this will be false, otherwise true.
  	\begin{lstlisting}[belowskip=0pt]
  while(myFile.good())
  	\end{lstlisting}
  \end{block}
\end{frame}

\begin{frame}[fragile]
  \frametitle{Introduction to \texttt{ifstream}}


	\begin{block}{Reading a line from the file}
  	\begin{lstlisting}[aboveskip=0pt]
ifstream & getline(istream & is, string & str, char delim);
ifstream & getline(istream & is, string & str);
  	\end{lstlisting}
  	This function reads characters from an input stream (\texttt{is}) until a delimiter (\texttt{delim}) is found.  The result is placed in the string \texttt{str}.  If you call it again it will continue from where it left off.
  	\newline\pause
  	The overloaded version with no third parameters assumes the delimiter to be \texttt{\textbackslash{}n} (i.e. the new line character).
  	\newline\pause
  	This is \textit{not} part of \texttt{ifstream}, it's actually in \texttt{<string>}.
  	\begin{lstlisting}[belowskip=0pt]
  std::getline(myFile, line);
  	\end{lstlisting}
  \end{block}
  
  These are just some of the many things \texttt{ifstream} can do, see \footnote{\url{http://www.cplusplus.com/reference/iostream/ifstream/}} for a complete look at the class.

\end{frame}

\begin{frame}[fragile]
	\frametitle{Input}
	\framesubtitle{Reading from a file}
	
	Let's have another look:
	\begin{columns}[t]
	  \begin{column}[T]{0.5\textwidth}
  		\lstinputlisting[language=C++,%
    		basicstyle=\ttfamily\fontsize{7}{8}\selectfont{},%
    		aboveskip=0pt,belowskip=0pt]%
    		{../code/4_io/lectures/simple_input.cpp}
		\end{column}
		\begin{column}[T]{0.5\textwidth}
		  Output:
			\begin{verbatim}Here, have a caterpillar:
						
   \
    '-.__.-'
    /oo |--.--,--,--.
    \_.-'._i__i__i_.'
          """""""""
		\end{verbatim}
		\end{column}
	\end{columns}
\end{frame}

\section{Strings revisited}

\begin{frame}[fragile]
  \frametitle{Strings revisited}
  \framesubtitle{How long is a piece of string? myString.length() of course}
  
  \texttt{string}s that we saw earlier are actually classes.  There are many useful things we can do with strings.  Here are a couple:
  \pause  
  \begin{block}{Get the length}
    \begin{lstlisting}[aboveskip=0pt]
  size_t size();
  size_t length();
    \end{lstlisting}
    Get the number of characters in a string.  \texttt{size\_t} is a special type that roughly corresponds to an unsigned integer.
  \end{block}
  \pause
  \begin{block}{Find content in the string}
    \begin{lstlisting}[aboveskip=0pt]
  size_t find(const string & str, size_t pos = 0) const;
    \end{lstlisting}
    Search for \texttt{str} within the current string, optionally starting after position \texttt{pos}, returning the position of the first occurrence in the string.  If \texttt{str} is not found \texttt{string::npos} is returned.
  \end{block}

\end{frame}

\begin{frame}[fragile]
  \frametitle{Splitting up a strings}
  \framesubtitle{Step 1}
  
  Say we have a string made up of a series of numbers:
	\lstinputlisting[language=C++,%
  	linerange={7-7}]%
  	{../code/4_io/lectures/parse_numbers.cpp}

  How do we get the numbers out so we can deal with them as \kw{double}s and, say, calculate their sum?\newline\pause
  By using the ever handy \texttt{stringstream} class:

  \begin{block}{Step 1: Split the string up using space as a delimiter}
		\lstinputlisting[language=C++,%
  		aboveskip=0pt,belowskip=0pt,
  		linerange={7-13}]%
  		{../code/4_io/lectures/parse_numbers.cpp}
	\end{block}

\end{frame}

\begin{frame}[fragile]
  \frametitle{Splitting up a strings}
  \framesubtitle{Step 2}
  
  \begin{block}{Step 2: Convert each number string to a double}
		\lstinputlisting[language=C++,%
  		aboveskip=0pt,belowskip=0pt,
  		linerange={13-19}]%
  		{../code/4_io/lectures/parse_numbers.cpp}
    Here we're using the extraction operator (\texttt{>>}) to extract a \kw{double} type from the \texttt{stringstream}.\pause{} \newline
    We can check if it failed by using the not operator (\texttt{!}).  Remember because this is a class operator we could write:
    \begin{lstlisting}
  numStream.operator >>(num);
    \end{lstlisting}
	\end{block}
	To use the \texttt{stringstream} class make just include the \texttt{<sstream>} header.

\end{frame}

\begin{frame}[fragile]
  \frametitle{Splitting up a string}
  %\framesubtitle{The whole shebang}
  \begin{columns}
    \begin{column}[T]{0.75\linewidth}
			\lstinputlisting[language=C++,%
			  aboveskip=0pt,belowskip=0pt,%
				basicstyle=\ttfamily\fontsize{8}{9}\selectfont{},%
  			title=\lstsrctitle]%
  			{../code/4_io/lectures/parse_numbers.cpp}
  	\end{column}
  	\begin{column}[T]{0.25\linewidth}
  	  \cout{Sum is: 18.5}
  	\end{column}
  \end{columns}  

\end{frame}

\end{document}