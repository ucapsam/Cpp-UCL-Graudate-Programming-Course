\documentclass{beamer}
%\documentclass[handout]{beamer}

% Input all common stuff


% THEME SETTINGS %%%%%%%%%%%%
\usetheme[secheader]{Boadilla}
\usecolortheme{default}
\useinnertheme{circles}
\setbeamertemplate{blocks}[default]

% Get rid of bottom navigation bars
\setbeamertemplate{footline}[page number]{}

% Gets rid of navigation symbols
\setbeamertemplate{navigation symbols}{}

% PACKAGES %%%%%%%%%%%%%%%%%%%
\usepackage{listings}
\usepackage{xcolor}
\usepackage{multirow}
\usepackage{array}
\usepackage{verbatim}
\usepackage{pbox}
\usepackage{tabularx}

% LISTINGS STYLE %%%%%%%%%%%%%
\definecolor{bluekeywords}{rgb}{0.13,0.13,1}
\definecolor{greencomments}{rgb}{0,0.5,0}
\definecolor{redstrings}{rgb}{0.9,0,0}
\definecolor{listingsbg}{HTML}{EFEFFF}

\lstset{language=C++,
  showspaces=false,            % show spaces adding particular underscores
  showtabs=false,              % show tabs within strings adding particular underscores
  breaklines=true,             % sets automatic line breaking
  showstringspaces=false,      % underline spaces within strings
  breakatwhitespace=true,      % sets if automatic breaks should only happen at whitespace
  captionpos=b,                % sets the caption-position to bottom
  escapeinside={(*@}{@*)},
  commentstyle=\color{greencomments},
  keywordstyle=\color{bluekeywords}\bfseries,
  stringstyle=\color{redstrings},
  basicstyle=\ttfamily\fontsize{9}{10}\selectfont,
  backgroundcolor=\color{listingsbg}
}

% MACROS %%%%%%%%%%%%%%%%%%%%%
\newcommand{\lstsrctitle}[0]{%
 {\scriptsize\lstname}
}

\newcommand{\kw}[1]{\texttt{\textcolor{bluekeywords}{\ttfamily\fontsize{9}{10}\selectfont #1}}}

\newenvironment{varblock}[3]{%
  \setbeamercolor{block body}{#2}
  \setbeamercolor{block title}{#3}
  \begin{block}{#1}}{\end{block}}

% Colours taken from http://twitter.github.com/bootstrap/components.html#alerts
% Do block %%%%%%%%%%%%%%%%%%%%%%
\definecolor{doblocktext}{HTML}{468847}
\definecolor{doblockbg}{HTML}{DFF0D8}
\newenvironment{doblocke}[0]{% begin
  \begin{varblock}{\textbf{Do}}%
  {bg=doblockbg,fg=doblocktext}{bg=doblocktext,fg=white}%
  \setbeamercolor{itemize item}{fg=doblocktext}
  \lstset{backgroundcolor=}% The background should be the same as that of the block
  } % begin
  {\end{varblock}} %end
  
\newcommand{\doblock}[1]{%
 \begin{doblocke}#1\end{doblocke} 
}

% Don't block %%%%%%%%%%%%%%%%%%%
\definecolor{dontblocktext}{HTML}{B94A48}
\definecolor{dontblockbg}{HTML}{F2DEDE}  
\newenvironment{dontblocke}[0]{% begin
  \begin{varblock}{\textbf{Don't}}%
  {bg=dontblockbg,fg=dontblocktext}{bg=dontblocktext,fg=white}
  \setbeamercolor{itemize item}{fg=dontblocktext}
  \lstset{backgroundcolor=}% The background should be the same as that of the block
  }%
  {\end{varblock}} %end

\newcommand{\dontblock}[1]{%
 \begin{dontblocke}#1\end{dontblocke} 
}

% Defi block %%%%%%%%%%%%%%%%%%%%
\definecolor{defiblocktext}{HTML}{3A87AD}
\definecolor{defiblockbg}{HTML}{D9EDF7}
\newenvironment{defiblocke}[1]{% begin
  \begin{varblock}{\textbf{Definition}}%
  {bg=defiblockbg,fg=defiblocktext}{bg=defiblocktext,fg=white}
  \lstset{backgroundcolor=}% The background should be the same as that of the block
  \textit{#1} }
  {\end{varblock}} %end

\newcommand{\defiblock}[2]{%
 \begin{varblock}{\textbf{Definition}}%
 {bg=defiblockbg,fg=defiblocktext}{bg=defiblocktext,fg=white}%
  \begin{tabularx}{\linewidth}{lX}\textit{#1} & #2\end{tabularx}
 \end{varblock} 
}

\newenvironment{defiblockbaree}[1]{ %begin
  \begin{varblock}{\textbf{#1}}%
  {bg=defiblockbg,fg=defiblocktext}{bg=defiblocktext,fg=white}
  \lstset{backgroundcolor=}} %begin
  {\end{varblock}} %end

\definecolor{warnblocktext}{HTML}{C09853}
\definecolor{warnblockbg}{HTML}{FCF8E3}
\newenvironment{warnblocke}[0]{% begin
  \begin{varblock}{\textbf{Warning!}}%
  {bg=warnblockbg,fg=warnblocktext}{bg=warnblocktext,fg=white}
  \lstset{backgroundcolor=}% The background should be the same as that of the block
  \setbeamercolor{itemize item}{fg=warnblocktext}
  \setbeamercolor{item projected}{fg=white,bg=warnblocktext}
  } % begin
  {\end{varblock}} % end

\newcommand{\warnblock}[1]{%
  \begin{warnblocke}#1\end{warnblocke}
} % newcommand

\newcommand{\cout}[1]{%
 Output: \pbox[t]{\textwidth}{\ttfamily\fontsize{9}{10}\selectfont{}#1}
}

\newenvironment{doitemize}[0]{%
  \begin{itemize}}%
  {\end{itemize}} %end


% CONSTANTS %%%%%%%%%%%%%%%%%%%%%%%%%%%

%Symbol for backslash in texttt \char`\\

% Common title slide stuff
\title{Introduction to Scientific Programming with C++}
\author{Martin Uhrin}
\institute{UCL}
\date{February 11-13th 2013}

\subtitle{Session 2: More data types}

\begin{document}

\frame{\titlepage}

\begin{frame}
\frametitle{Table of Contents}
\tableofcontents
\end{frame}


\section{Arrays}

\subsection{Declaring arrays}

\begin{frame}[fragile]
	\frametitle{Arrays}
	
	\defiblock{array}{a series of elements of the same type occupying a contiguous block of memory.}
	
	Format for declaring an array is:
	\begin{lstlisting}
	type name[num_elements];
	\end{lstlisting}
	Where \texttt{type} is any valid data type and \texttt{num\_elements} is a constant positive integer.
	\pause
	Some examples:
	\begin{lstlisting}
	unsigned int lotteryNumbers[7];
	
	double planetMasses[8];
	
	const unsigned int numParticles = 128;
	double xPositions[numParticles];
	double yPositions[numParticles];
	\end{lstlisting}
	Last example shows how we can use constant variable as array size.
	
\end{frame}

\subsection{Using arrays}

\begin{frame}[fragile]
  \frametitle{Initialising arrays}
  When declaring an array it can be initialised as follows:
  \begin{lstlisting}
unsigned int lotteryNumbers[7] = {16, 3, 28, 9, 24, 10, 8}
  \end{lstlisting}
  the size can be left out, in which case the number of values given is used:
  \begin{lstlisting}
unsigned int lotteryNumbers[] = {16, 3, 28, 9, 24, 10, 8}
  \end{lstlisting}  
\end{frame}

\begin{frame}[fragile]
  \frametitle{Accessing elements}
  
  To access an element of an array the format is:
  \begin{lstlisting}
  name[index]
  \end{lstlisting}
  \warnblock{In C++ array numbering starts at 0!  This is a huge source of confusion especially if you're used to a programming language like Fortran where arrays start at 1.}
  \pause
  For example, to read the 3$^{rd}$ lottery number use:
  \begin{lstlisting}
  unsigned int third = lotteryNumbers[2];
  \end{lstlisting}
  \pause
  To write the 3$^{rd}$ lottery number use:
  \begin{lstlisting}
  lotteryNumber[2] = 23;
  \end{lstlisting}
  \pause
  You can also access the elements using a variable:
  \begin{lstlisting}
  for(int i = 0; i < 7; ++i)
     std::cout << lotteryNumbers[i] << " ";
  \end{lstlisting}
\end{frame}

\begin{frame}[fragile]
  \frametitle{Accessing elements}
  \framesubtitle{Array index out of bounds}
    \begin{warnblocke}Be careful not to access elements past the end of an array.  Consider:
    \begin{lstlisting}
  const unsigned int numPlanets = 8;
  double masses[numPlanets];
  for(int i = 0; i <= numPlanets; ++i)
    masses[i] = random();
    \end{lstlisting}
	Seems ok, but wait.  The last iteration will try to set \texttt{masses[8]} which is past the end of the array!
    \end{warnblocke}

\end{frame}

\subsection{Multidimensional arrays}

\begin{frame}[fragile]
  \frametitle{Multidimensional arrays}
  \framesubtitle{Because with only two friends 1D is sooo boring}
  Think of multidimensional arrays as being "arrays of arrays".  An example:
  \lstinputlisting[language=C++,title=\lstsrctitle,linerange={11-16,23-33}]{../code/2_more_data_types/lectures/centre_of_mass.cpp}  
  

\end{frame}

\begin{frame}[fragile]
  \frametitle{Multidimensional arrays}
  \framesubtitle{Why stop at two}
  In theory you can have as many array dimensions as you want.  For example a 3D array of ising spins could be represented as:
  \begin{lstlisting}
  bool isingSpins[nX][nY][nZ];
  \end{lstlisting}
  In practice you have to worry about how much memory your array needs!\pause{}  If you want to find out use:
  \begin{lstlisting}
  std::cout << "Need: " << sizeof(bool) * nX * nY * nZ <<
    " bytes";
  \end{lstlisting}

\end{frame}

\begin{frame}[fragile]
	\frametitle{std::array}
	\framesubtitle{Like c-style arrays but safer and easier to use}
	\begin{lstlisting}
std::array<double,5> myArray={2.3, 4.2, 7.5, 8.2, 9.1};
	\end{lstlisting}
	These arrays are just as fast, have built in protection and are easier to manipulate. They protect against code such as this:
	\begin{lstlisting}
double SingleValue = myArray[8];
	\end{lstlisting}
	Accessing elements outside a C-style array is dangerous.\\
	To use them the following header file must be included
	\begin{lstlisting}
#include<array>
	\end{lstlisting}
\end{frame}

\section{Char sequences and strings}


\begin{frame}[fragile]
  \frametitle{Strings}
  \framesubtitle{A string of characters, used to store text.}
  
  \begin{lstlisting}
std::string message = "Physics rocks!";
  \end{lstlisting}
  Strings use the c-style char arrays internally but are much easier to use.
  To use strings make sure to use the following include:
  \begin{lstlisting}
#include <string>  // At the top of your file
  \end{lstlisting}
\end{frame}

\begin{frame}[fragile]
  \frametitle{String variables}
  \framesubtitle{All strung out}
  So what can we do with strings?
	
  \begin{block}{Initialise}
      \begin{lstlisting}
std::string firstName = "Bjarne";
std::string lastName("Stroustrup"); // Equivalent to above
      \end{lstlisting}
  \end{block}
  \pause
  \begin{block}{Concatenate (add two or more together)}
      \begin{lstlisting}
std::string fullName = firstName + " " + lastName;
      \end{lstlisting}
			Notice ability to mix string variables and literals (i.e. things in quotes).      
  \end{block}
  \pause
  \begin{block}{Read from user}
    \begin{lstlisting}
std::cout << "Enter first name: "
std::cin >> firstName;
    \end{lstlisting}
  \end{block}
\end{frame}

\section{Pointers and references}

\subsection{Pointers}

\begin{frame}[fragile]
  \frametitle{Pointers}
  \framesubtitle{Pointers are low level C-stlye code that can be very dangerous, they have literally cost lives. In modern C++ they can usually be avoided.}
  Every variable lives at a memory address.  A pointer is a special data type that stores such an addresses.
  \begin{block}{Pointer declaration}
  To declare a pointer the format is:
	  \begin{lstlisting}
  type * name;
	  \end{lstlisting}
	  This tells the compiler that \texttt{name} is a pointer that points to the address of a variable of type \texttt{type}.  Got it?
	  \pause\newline
	  Some examples:
	  \begin{lstlisting}[belowskip=0pt]
  int * pointerToInt;
  std::string * pointerToString;
	  \end{lstlisting}
  \end{block}
\end{frame}


\begin{frame}[fragile]
  \frametitle{Using pointers}
  \begin{block}{Pointer initialisation}
	  To set pointers we can use the reference operator: \texttt{\&} (read "address of").
	\lstinputlisting[language=C++,linerange={5-5,7-7}]{../code/2_more_data_types/lectures/pointers_example.cpp}  line 2 tells the compiler to:
	  \begin{enumerate}
	    \pause
	    \item{Create a pointer named \texttt{spinsPointer} that points to an \kw{int}.}
	    \pause
	    \item{Set it to address of \texttt{upSpins}.}
	  \end{enumerate}
  \end{block}
  \pause
  \begin{block}{So what's in a pointer?}
	  \begin{columns}[t]
	    \begin{column}[T]{0.5\textwidth}
			  \lstinputlisting[language=C++,firstline=9,lastline=11,aboveskip=0pt,belowskip=0pt]{../code/2_more_data_types/lectures/pointers_example.cpp}
			\end{column}\pause
			\begin{column}[T]{0.5\textwidth}
			  \cout{Address is: 0x00CBF748}\newline
			  This is how C++ prints memory addresses.
	    \end{column}
	  \end{columns}
  \end{block}
\end{frame}

\begin{frame}[fragile]
  \frametitle{Using pointers}
  
  \begin{block}{Reading the value}
  	To access, use the the dereference operator: \texttt{*} (read "value pointed by").
  	\begin{columns}[t]
  	  \begin{column}[T]{0.5\textwidth}
 \lstinputlisting[language=C++,firstline=13,lastline=14,aboveskip=0pt,belowskip=0pt]{../code/2_more_data_types/lectures/pointers_example.cpp}
  	  \end{column}\pause
  	  \begin{column}[T]{0.5\textwidth}
  	    \cout{Value is: 10}
  	  \end{column}
  	\end{columns}
  \end{block}
  \pause
  \begin{block}{Setting the value}
    To set the value pointed to by a pointer also use dereference operator:
  	\begin{columns}[t]
  	  \begin{column}[T]{0.5\textwidth}
\lstinputlisting[language=C++,firstline=17,lastline=19,aboveskip=0pt,belowskip=0pt]{../code/2_more_data_types/lectures/pointers_example.cpp}  	  \end{column}\pause
  	  \begin{column}[T]{0.5\textwidth}
  	    \cout{New upSpins: 20}
  	  \end{column}
  	\end{columns}
  \end{block}
  \pause
  \begin{block}{Assigning pointers}
  Later on we may want to assign the pointer to point to a different value:
  	\begin{columns}[t]
  	  \begin{column}[T]{0.5\textwidth}
\lstinputlisting[language=C++,linerange={6-6,22-24},aboveskip=0pt,belowskip=0pt]{../code/2_more_data_types/lectures/pointers_example.cpp}
  		\end{column}\pause
  	  \begin{column}[T]{0.5\textwidth}
  	    \cout{Down spins: 7}
  	  \end{column}
  	\end{columns}
  \end{block}
\end{frame}

\begin{frame}[fragile]
  \frametitle{Pointers}
  \framesubtitle{More dangerous than a global warming and nuclear weapons combined}
  \begin{warnblocke}
    C++ pointers can be dangerous!  Consider:
    \begin{lstlisting}
  int * upSpinsPointer;
  std::cout << *upSpinsPointer;
    \end{lstlisting}
    I've asked for the value pointed by \texttt{upSpinsPointer}.  But what's it pointing to?  It could be a valid address or garbage.
  \end{warnblocke}
  \pause
  \begin{doblocke}
    Set pointers to the special value \texttt{0} upon declaration which indicates that it is not pointing to a valid memory address:
    \begin{lstlisting}
  int * upSpinsPointer = 0;
    \end{lstlisting}
    This is called a null pointer.  The program will crash immediately if you try to dereference it and you will find out straight away what went wrong.
  \end{doblocke}

\end{frame}

\begin{frame}[fragile]
  \frametitle{Dynamic memory}
  So far our programs have used a fixed amount of memory equal to the total of all declared variables.  Sometimes we don't know how much memory we will need before the program starts.
  \pause
  Consider:
  \begin{lstlisting}
unsigned int spinChainLength;
std::cout << "Enter spin chain length: ";
std::cin >> spinChainLength;
bool spinChain[spinChainLength]; // ERROR!
  \end{lstlisting}
  But we can only create arrays of known, \textit{constant}, size!
  \newline\pause
  Solution: dynamic memory.
  \begin{block}{\kw{new} and \texttt{\kw{new}[]} operators}
	  To allocate dynamic memory the format is:
	  \begin{lstlisting}
  pointer = new type;               // single variable
  pointer = new type[num_elements]; // array
	  \end{lstlisting}
  \end{block}
  \pause
  Let's try again:
  \begin{lstlisting}
bool * spinChain = new bool[spinChainLength]; // Good
  \end{lstlisting}
\end{frame}

\begin{frame}[fragile]
  \frametitle{Dynamic memory}
  \framesubtitle{Don't forget to clean up}
  
  \begin{block}{\kw{delete} and \texttt{\kw{delete}[]} operators}
    To free dynamic memory the format is:
    \begin{lstlisting}
  delete pointer;   // single variable
  delete[] pointer; // array
    \end{lstlisting}
    Make sure you use the correct version, otherwise bad things \textit{will} happen.
  \end{block}
  \pause
  \dontblock{Forget the free your memory once you're done with it.  Otherwise it will leak and you won't get it back until your program ends!}
  \pause
  \begin{doblocke}
  Set the pointer to \texttt{0} after the memory has been freed:
  \begin{lstlisting}[belowskip=0pt]
  delete[] spinChain;
  spinChain = 0;
  \end{lstlisting}
  \end{doblocke}
\end{frame}

\subsection{References}

\begin{frame}[fragile]
  \frametitle{References}
  \framesubtitle{More safe than keeping your money in a Swiss bank account}
  
  A reference is similar to a pointer only more limited, and more safe.
  \begin{block}{Reference declaration and initialisation}
	  To declare a reference and initialise it the format is:
		  \begin{lstlisting}
  type & name = variable_name;
		  \end{lstlisting}
	  This tells the compiler that name is a reference to an existing variable called \texttt{variable\_name} which is of type \texttt{type}.  References \textit{cannot} be uninitialised!\newline
		Example:
    \lstinputlisting[language=C++,linerange={5-5,7-7},aboveskip=0pt,belowskip=0pt]%
      {../code/2_more_data_types/lectures/references_example.cpp}
    \begin{lstlisting}[aboveskip=0pt,belowskip=0pt]
  int & downSpins; // Error: cannot be uninitialised
    \end{lstlisting}  
  \end{block}
  \pause
  \begin{block}{Using references}
  	Once a reference is declared it can be used almost exactly as if it were an ordinary variable.
  \end{block}

\end{frame}

\begin{frame}
  \frametitle{Pointers vs. references}
  \framesubtitle{Practical comparison}
  
  	\begin{columns}[t]
  	  \begin{column}[T]{0.5\textwidth}
\lstinputlisting[language=C++,aboveskip=0pt,belowskip=0pt,basicstyle=\ttfamily\fontsize{7}{8}\selectfont]{../code/2_more_data_types/lectures/pointers_example.cpp}
			\end{column}
  	  \begin{column}[T]{0.5\textwidth}
  	    \lstinputlisting[language=C++,aboveskip=0pt,belowskip=0pt,basicstyle=\ttfamily\fontsize{7}{8}\selectfont]{../code/2_more_data_types/lectures/references_example.cpp}
  	  \end{column}
  	\end{columns}
  
\end{frame}

\section{Functions revisited}

\subsection{Reference and pointer parameters}


\begin{frame}[fragile]
  \frametitle{Changing variable value in a function}
  Consider:
  \begin{columns}[t]
    \begin{column}[T]{0.6\textwidth}
	    \begin{lstlisting}[aboveskip=0pt]
void runningSum(int sum, int value)
{
  sum += value;
}
int main()
{
  int sum = 0;
  for(int i = 1; i < 100; ++i)
    runningSum(sum, i);
    
  std::cout << "Sum is: "
    << sum << "\n";
}
      \end{lstlisting}
    \end{column}
    \begin{column}[T]{0.4\textwidth}
      \cout{Sum is: 0}
    \end{column}
  \end{columns}
  But wait.  What's happened? \texttt{sum} is still 0\footnote{Although our program fails to calculate this sum, legend has it that Gauss produced the correct answer within seconds when asked during a primary school lesson.}.
  \newline\pause
  Problem is that \texttt{runningSum} got a \textit{copy} of the value in \texttt{sum} at the time it was called.

\end{frame}

\begin{frame}[fragile]
  \frametitle{Passing using pointers}
  
  Let's try again but this time using a pointer:
  \begin{columns}[t]
    \begin{column}[T]{0.6\textwidth}
	    \begin{lstlisting}[aboveskip=0pt]
void runningSum(int * sum, int value)
{
  *sum += value;
}
int main()
{
  int sum = 0;
  for(int i = 1; i <= 100; ++i)
    runningSum(&sum, i);
    
  std::cout << "Sum is: "
    << sum << "\n";
}
      \end{lstlisting}
    \end{column}
    \begin{column}[T]{0.4\textwidth}
      \cout{Sum is: 5050}
    \end{column}
  \end{columns}
  
\end{frame}

\begin{frame}[fragile]
  \frametitle{Passing using references}
  
  Let's try again but this time using a references:
  \begin{columns}[t]
    \begin{column}[T]{0.6\textwidth}
	    \begin{lstlisting}[aboveskip=0pt]
void runningSum(int & sum, int value)
{
  sum += value;
}
int main()
{
  int sum = 0;
  for(int i = 1; i <= 100; ++i)
    runningSum(sum, i);
    
  std::cout << "Sum is: "
    << sum << "\n";
}
      \end{lstlisting}
    \end{column}
    \begin{column}[T]{0.4\textwidth}
      \cout{Sum is: 5050}
    \end{column}
  \end{columns}
  
\end{frame}

\begin{frame}[fragile]
  \frametitle{Passing by value, pointer and reference}
  \framesubtitle{What's the difference?}
  
  So we have three ways to pass parameters:
  \begin{lstlisting}
/*1.*/ void runningSum(int sum, int value);  // by value
/*2.*/ void runningSum(int * sum, int value);// by pointer
/*3.*/ void runningSum(int & sum, int value);// by reference
  \end{lstlisting}
  Imagine you're the function and I'm passing you an Edward Hopper\footnote{Nighthawks to be exact (\url{http://en.wikipedia.org/wiki/Nighthawks})} painting I keep on my living room wall:
  \begin{enumerate}
  	\pause
    \item{I make an exact duplicate and give it to you.  Any changes you make to yours don't affect mine.}
    \pause
    \item{I give you my address and you can view and change the painting by visiting (dereferencing) my address.}
    \pause
  	\item{I create a second painting that is quantum entangled with mine.  Any changes you make to yours affect mine instantly.}
  \end{enumerate}

\end{frame}


\end{document}